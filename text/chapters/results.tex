% !TEX root = ../main.tex
\section{Results}
\countem

\subsection{Identification of a possible novel prognostic marker}

In a multivariate Cox analysis in 1218 breast cancer patients, I could identify 2078 significantly ($q < 0.05$) associated genes (mRNA expression) with the overall survival.
SPIB was associated with a beneficial hazard ratio ($HR = 0.91$; $q = 3.27\%$; Fig.~\ref{km_plot}).
Further, patients overexpressing SPIB have an overall median survival of 130 months, while low expressing patients survive for a median of 112 months (logrank: $p = 0.01$; Fig.~\ref{km_plot}).
Five years overall survival was 77\% (SPIB-low) and 81.4\% (SPIB-high) respectively. 

\subsection{Methylation and mRNA Expression}

Next, I focussed on the regulation of SPIB through methylation.
From the selected patient cohort, 873 samples had mRNA sequencing and methylation data available.  
A total of 17 CpG islands associated with SPIB were analysed (Table~\ref{meth_table}).
In the spearman correlation site cg07979271 displayed the most significant negative association between methylation and SPIB mRNA expression levels ($cor = 0.55; q = 9.43 \cdot 10^{-59}$; Fig.~\ref{methylation}).

\subsection{Copy Number Alteration and mRNA Expression}

Another way mRNA expression could be modified is through copy number alteration.
A subset of 1078 samples included mRNA sequencing and copy number data.
As seen in Fig.~\ref{cnv}, no significant difference between the copy number levels could be shown (one-way independent ANOVA; $p > 0.05$).

\subsection{Co-Expression Analysis}

As SPIB is a known transcription factor, the co-expressed genes are of major interest.
The mRNA data of 1218 samples was available and the most significant correlations are shown in Table~\ref{mrna_table}.
Many co-expressed genes are involved in immunological pathways, MS4A1 displayed the strongest positive correlation and encodes the CD20 antigen found on B-lymphocytes ($cor = 0.82; q = 2.98 \cdot 10^{-295}$).
Further genes include TCL1A,CXCR5,CCL4 (found on T-cells), LCK (found on lymphocytes in general) or MAP4K1 (a JNK-activator).
Negative correlations were also investigated (Table~\ref{mrna_table}) with DCTN4 displaying the strongest one ($cor = -0.42; q = 4.77 \cdot 10^{-51}$).
\\
\endcountem 
(Results word count: \thewordcount{})