% !TEX root = ../main.tex
\section{Introduction}
\countem

Breast cancer is the most common cancer in women.
One in every seven women in the UK will be diagnosed with breast cancer in their lifetime\cite{cancerresearchuk_2021}. 
Due to the high incidence, breast cancer is also accountable for the most cancer-associated deaths.
To predict patient outcomes and to find new actionable targets for cancer therapy new biomarkers are required.
There is already a multitude of established clinical markers like the tumour size, lymphatic node invasion status, distant metastasis status (as in the TNM classification) and molecular characterization (mainly oestrogen, progesterone and HER2 receptor status).
High throughput data is becoming more readily available, allowing for in silico analysis of multiomics (e.g., genomics, proteomics, \ldots) of large cohorts.
This essay tries to demonstrate the utility of SPIB as a possible breast cancer biomarker.
The transcription factor Spi-B is encoded by the gene SPIB and is a member of the Erythroblast Transformation Specific (ETS) group, which is defined by a common highly conserved DNA-binding domain.
In the literature SPIB is described as both a tumour suppressor and an oncogenic protein.
Studies in lung cancer cells found SPIB to be involved in recruitment of tumour associate macrophages (TAM) \cite{Huang2021}, further SPIB was found to promotes anoikis resistance \cite{Zhang2020}.
In colorectal cancer cells, on the other hand, SPIB displayed tumour suppressing characteristics by activating NF-$\kappa$B and JNK signalling pathways \cite{Zhao2021}.
\\
\endcountem 
(Introduction word count: \thewordcount{})