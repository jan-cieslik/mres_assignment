% !TEX root = ../main.tex
\section{Conclusion}
\countem

SPIB is an interesting gene that is currently under explored in breast cancer.
It could already be demonstrated to be a viable candidate gene in lung \cite{Huang2021, Zhang2020} and colorectal \cite{Zhao2021} cancer.
This essay shows a possible positive association of SPIB with survival in breast cancer.
Overexpression of SPIB increases the median overall survival in breast cancer by 18 months (112 vs 130 months).
The initial regulation analysis found cg07979271 to be the highest anti-correlated CpG site and no association between copy number alterations.
Indicating a strong transcriptional regulation of SPIB as an increase in copy number does not increase its expression.
As already found in different cancer entities \cite{Huang2021,Zhang2020} SPIB mRNA expression has a positive correlation with immunologic pathways on B- and T-cells (e.g., CCL4).
In lung cancer this seems to induce an increased recruitment of tumour\-/associated macrophages \cite{Huang2021} leading to a shortened overall survival.
The data in colorectal cancer,\ on\ the\ other\ hand, suggest SPIB to act as a tumour suppressor by activating NF$\kappa$B and JNK signalling \cite{Zhang2020}.
Both effects could be demonstrated in this analysis during correlation of mRNA expression of the selected genes and SPIB.
As this is purely a correlation and SPIB is a known transcription factor further studies are required to understand the involvement of SPIB in breast cancer.
One limitation of this work is the non\-/directional correlation, as it is not clear if SPIB is the protein affecting the others or if the opposite is true.
\\
\endcountem 
(Conclusion word count: \thewordcount{})