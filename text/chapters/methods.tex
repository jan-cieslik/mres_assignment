% !TEX root = ../main.tex
\section{Methods}
\countem

\subsection{Data Sources}

The Cancer Genome Atlas (TCGA) dataset for breast cancer (BRCA)\cite{Ciriello2015} was acquired from the Xena platform \cite{Goldman2018}.

\subsection{Data Pre-Processing and Normalization}

Some data was downloaded after pre-processing steps have already been performed.
I will outline all important data pre-processing steps to give a complete representation of the resulting data.

\subsubsection{RNA-Seq}

Gene expression profiles from the Illumina HiSeq 2000 RNA sequencing platform were aligned and annotated using reference transcripts based on the hg19 reference genome.
Transcript abundance was calculated with RSEM (RNA-Seq by Expectation Maximization) and transformed to log2(x+1) normalized values.

\subsubsection{Methylation}

Methylation data from the Illumina Infinium HumanMethylation450 BeadChip (Methylation450k) was transformed into $\beta$ values ranging from 0 to 1.
A higher $\beta$ value indicates a higher level of DNA methylation.
The probes were aligned to the hg19 reference genome.
Methylation probes were then annotated with their corresponding gene symbol.

\subsubsection{Copy Number}

Copy number data was obtained via a whole genome microarray.
The raw data was processing using the GISTIC2 (Genomic Identification of Significant Targets in Cancer) pipeline.
Finally, the resulting values were grouped by thresholds and transformed into one of five levels from deep deletion (-2) to high-level amplification (+2).

\subsection{Identification of a Possible Novel Prognostic Marker}

A multivariate Cox proportional hazards regression model was created with data from 1218 breast cancer patients.
Overall survival was defined as the dependent (outcome) variable while mRNA log values (as determined by RNA sequencing from the primary tumour) together with age and tumour stage (I/II as low; III/IV as high) were defined as the independent variables.
The calculated p-values were corrected with the false discovery rate (FDR) method and are shown as q-values.

\subsection{Survival Analysis}

The patient population was divided into SPIB-high and SPIB-low based on the median of the SPIB mRNA expression.
A logrank test and an associated Kaplan-Meier plot were generated for the two subpopulations.

\subsection{Methylation and mRNA Expression Correlation}

Methylation values were tested for normal distribution using Shapiro-Wilk normality test.
The $\beta$ values fail to show a normal distribution (e.g., cg07979271: ${p < 2.2 \cdot 10^{-16}}$), consequently a nonparametric test was used for correlation.
Methylation $\beta$ values were correlated against the mRNA log values using spearman.
P-values were corrected for multiple testing by using FDR and transformed into q-values.
The most anti-correlated methylation site is then shown as a scatter plot against the mRNA level.

\subsection{Copy Number and mRNA Expression}

A chart with multiple box plots (one per copy number threshold) was generated displaying the corresponding mRNA log values.
To test for dependence of mRNA expression on copy number I performed a one-way independent ANOVA (analysis of variance).

\subsection{mRNA Co-Expression}

mRNA expression values were correlated with the corresponding expression values of SPIB using spearman.
The p-values were adjusted for multiple testing by using FDR and transformed into q-values.
\\
\endcountem
(Methods word count: \thewordcount{})